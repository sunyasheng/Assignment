\documentclass[10pt,twocolumn,letterpaper]{article}

    \usepackage{cvpr}
    \usepackage{times}
    \usepackage{epsfig}
    \usepackage{graphicx}
    \usepackage{amsmath}
    \usepackage{amssymb}
    
    % Include other packages here, before hyperref.
    
    % If you comment hyperref and then uncomment it, you should delete
    % egpaper.aux before re-running latex.  (Or just hit 'q' on the first latex
    % run, let it finish, and you should be clear).
    \usepackage[pagebackref=true,breaklinks=true,letterpaper=true,colorlinks,bookmarks=false]{hyperref}
    
    \cvprfinalcopy % *** Uncomment this line for the final submission
    
    \def\cvprPaperID{****} % *** Enter the CVPR Paper ID here
    \def\httilde{\mbox{\tt\raisebox{-.5ex}{\symbol{126}}}}
    
    % Pages are numbered in submission mode, and unnumbered in camera-ready
    \ifcvprfinal\pagestyle{empty}\fi
    \begin{document}
    
    %%%%%%%%% TITLE
    \title{Technical Report for Third Project in Parallel Computing}
    
    \author{Yasheng Sun\\
    117020910076\\
    {\tt\small sunyasheng123@gmail.com}
    % For a paper whose authors are all at the same institution,
    % omit the following lines up until the closing ``}''.
    % Additional authors and addresses can be added with ``\and'',
    % just like the second author.
    % To save space, use either the email address or home page, not both
    }
    
    \maketitle
    %\thispagestyle{empty}
    
    %%%%%%%%% ABSTRACT
    \begin{abstract}
       This report concludes the third project in parallel computing. This projcet target
       at implementing word counting pragram using MapReduce. This is a pretty elementry task
       in MapReduce Framework, which follows the classical MapReduce Pattern-Summarization Pattern.
        We will explain whole workflow briefly in this report. 
       All my code is publicly available on my github
        \url{https://github.com/sunyasheng/Parallel-Computing}.

    \end{abstract}
    
    %%%%%%%%% BODY TEXT
    
    
    %-------------------------------------------------------------------------
    \section{Data Preparation}
    
    In this project, we choose the novel Gone with The wind as the system input as is shown
    in Fig.~\ref{gone_with_the_wind}. 
    This novel contains 2657422 characters including Chinese, English and Punctuation.
    Mixture of various types of languages and characters aims to mimic the true scenario
    in real world.

    \section{Methodology}

    \begin{figure}[t]
        \begin{center}
        % \fbox{\rule{0pt}{2in} \rule{0.9\linewidth}{0pt}}
           \includegraphics[width=0.7\linewidth]{gone_with_the_wind.png}
        \end{center}
           \caption{Gone with the Wind}
        \label{gone_with_the_wind}
        \label{fig:long}
        \label{fig:onecol}
    \end{figure}
    In this section, the whole workflow of MapReduce is illustrated in Fig.~\ref{workflow}. 
    The task of Mapper is to select every English item from orignal article and map
    the corresponding item to 1. Then the output stream will be
    shuffled, sorted and passed to Reducer. Reducer will go through all
    the items and add up all the numbers for each item. 

    Note that all English items have to be transported from the Mapper Node to Reducer Node,
    which implies that if a certain English item will be transported mutiple times if it is 
    repeated in an article many times. 
    The inter-node transportation is usually very
    expensive. To lower the amount of transportation, A practical technique Combiner is introduced
    here to add up the number of items in its current Mapper Node.
    The Combiner is essentially a lcoal reducer. With this pre-reduction, 
    the transportation amount between different nodes can be decreased significantly.
    
    \section{Result}
    
    Only part output of this mapreduce task is shown in Fig.~\ref{output} because there are too 
    many words to be completely presented here. The word and their frequency of exisitence are
    listed row by row. 
    \begin{figure}[t]
        \begin{center}
        % \fbox{\rule{0pt}{2in} \rule{0.9\linewidth}{0pt}}
           \includegraphics[width=1.0\linewidth]{workflow.png}
        \end{center}
           \caption{\label{workflow}The MapReduce workflow.}
        \label{fig:long}
        \label{fig:onecol}
    \end{figure}

    \begin{figure}[t]
        \begin{center}
        % \fbox{\rule{0pt}{2in} \rule{0.9\linewidth}{0pt}}
           \includegraphics[width=1.0\linewidth]{wordcount_output.png}
        \end{center}
           \caption{System output for word count program.}
        \label{output}
        \label{fig:long}
        \label{fig:onecol}
    \end{figure}
    
    From Fig.~\ref{wordcount_info} we could know that the mapper task is 
    split to two nodes and the output stream will be passed to one reduce 
    node to finally summarize the result after map operation.
    \begin{figure}[t]
        \begin{center}
        % \fbox{\rule{0pt}{2in} \rule{0.9\linewidth}{0pt}}
           \includegraphics[width=1.0\linewidth]{wordcount_info.png}
        \end{center}
           \caption{System information in MapReduce Task.}
        \label{wordcount_info}
        \label{fig:long}
        \label{fig:onecol}
    \end{figure}

    Fig.~\ref{with_combiner} shows that the Combiner technique significantly
    reduces the amount of reduce input records and reduce shuffle bytes compared
    with the traditional routine without Combiner as is shwon in Fig.~\ref{without_combiner}.
    This improvement will play a more important role when it comes to a larger scale
    dataset.

    \section{Conclusion}
    A sample word count program is presented in this project, by which we show the
     MapReduce paradigme -- Summarization Pattern. A practical technique, 
    Combiner, is introduced here to lower the transportation amount between 
    Mapper Node and Reducer Node. Future work will involve more complicated problems
    in large scale.
    \begin{figure}[t]
        \begin{center}
        % \fbox{\rule{0pt}{2in} \rule{0.9\linewidth}{0pt}}
           \includegraphics[width=1.0\linewidth]{with_combiner.png}
        \end{center}
           \caption{The reduce input records and shuffle bytes information with Combiner.}
        \label{fig:long}
        \label{fig:onecol}
        \label{with_combiner}
    \end{figure}    
    \begin{figure}[t]
        \begin{center}
        % \fbox{\rule{0pt}{2in} \rule{0.9\linewidth}{0pt}}
           \includegraphics[width=1.0\linewidth]{without_combiner.png}
        \end{center}
           \caption{The reduce input records and shuffle bytes information without Combiner.}
        \label{fig:long}
        \label{fig:onecol}
        \label{without_combiner}
    \end{figure}    
    
    {\small
    \bibliographystyle{ieee}
    \bibliography{egbib}
    }
    \begin{thebibliography}{1}

    \bibitem{IEEEhowto:kopka}
    % H.~Kopka and P.~W. Daly, \emph{A Guide to \LaTeX}, 3rd~ed.\hskip 1em plus
    % 0.5em minus 0.4em\relax Harlow, England: Addison-Wesley, 1999.
    % D. Pathak, R. Girshick, P. Doll ́ ar, T. Darrell, and B. Hariha-ran.
    % Learning features by watching objects move. In \emph{CVPR}, 2017.
    \url{http://sunyasheng.github.io/}
    \end{thebibliography}

    \end{document}