\documentclass[10pt,twocolumn,letterpaper]{article}

    \usepackage{cvpr}
    \usepackage{times}
    \usepackage{epsfig}
    \usepackage{graphicx}
    \usepackage{amsmath}
    \usepackage{amssymb}
    
    % Include other packages here, before hyperref.
    
    % If you comment hyperref and then uncomment it, you should delete
    % egpaper.aux before re-running latex.  (Or just hit 'q' on the first latex
    % run, let it finish, and you should be clear).
    \usepackage[pagebackref=true,breaklinks=true,letterpaper=true,colorlinks,bookmarks=false]{hyperref}
    
    \cvprfinalcopy % *** Uncomment this line for the final submission
    
    \def\cvprPaperID{****} % *** Enter the CVPR Paper ID here
    \def\httilde{\mbox{\tt\raisebox{-.5ex}{\symbol{126}}}}
    
    % Pages are numbered in submission mode, and unnumbered in camera-ready
    \ifcvprfinal\pagestyle{empty}\fi
    \begin{document}
    
    %%%%%%%%% TITLE
    \title{Technical Report for Second Project in Parallel Computing}
    
    \author{Yasheng Sun\\
    117020910076\\
    {\tt\small sunyasheng123@gmail.com}
    % For a paper whose authors are all at the same institution,
    % omit the following lines up until the closing ``}''.
    % Additional authors and addresses can be added with ``\and'',
    % just like the second author.
    % To save space, use either the email address or home page, not both
    }
    
    \maketitle
    %\thispagestyle{empty}
    
    %%%%%%%%% ABSTRACT
    \begin{abstract}
       This report concludes the second project in parallel computing. 
       This project involves the basic usage of OpenMP function and 
       application of OpenMP in real world. We achieve the 
       approximation of $\pi$ following the previous project but through
       integration under OpenMP framework. Furthermore, we explore the
       usage of OpenMP in some advanced data structure such as link
       list instead of naive array data structure. All my code is publicly available on my github 
       \url{https://github.com/sunyasheng/Parallel-Computing}.

    \end{abstract}
    
    \section{Introduction}
    Project involves the understanding and application of OpenMP. In this project, we first put
    forward a more elegant way to compute $\pi$ through integration than the method in project 1. 
    Then we focus on the application of OpenMP to a more complicated data structure such as link list
    instead of naive array form[1]. A simple but tedious method is first proposed and then we refine 
    the code based on some advanced characteristic such as Task in OpenMP.
    
    %-------------------------------------------------------------------------
    \section{Estimate $\pi$ through integration}
    \subsection{Methodology}
    In previous project we estimate $\pi$ by Mento Carto which is not very accurate. Here we 
    estimate $\pi$ based on integration. 
    $$\int_0^1 \frac{1}{1 + x^2} dx = \frac{\pi}{4}$$
    This integration is computed by summing up all the areas of rectangles below the function curve.
    More rectangles are used, more precise the estiamted integration is. 
    
    From Figure \ref{sample_code} we can find that usage of OpenMP is
    very elegant and concise. We only need to note where we want to 
    parallel and what we want to reduce and everything will be finished
    by the complier. It is pretty convenient that less code is needed to do reduction in a neat fashion
    under the OpenMP framework. 

    
    \subsection{Result}
    Here we present the computation result under different number of threads when
    we divide the integration interval to 1000000000 segments in the Table~\ref{estimation_performance}.
    The estimated value is pretty close to the ground truth value. This 
    table shows us the power of parallelization in efficiency improvement. By using 
    two threads, we almost obtain double productivity compared with the traditional
    serial programing pattern. In the meantime, the complier have done the reduction
     automatically for us, which simplifies the programing significantly.


    %%%%%%%%% BODY TEXT
    \section{OpenMP Parallization in Link List}
    So far, the application of OpenMP we illustrated is limited in simple data 
    structure such as array or matrix. More complicated data structure is required
    in real world application to describe and store data.
    \subsection{Problem Statement}
    We formulate this problem as shown in Figure~\ref{problem_statement}. Many tasks
    is stored by a link list, which brings difficulty to direct parallelization under
    OpenMP framework. 
    \begin{figure}[t]
        \begin{center}
           \includegraphics[width=1.\linewidth]{images/sample_code.png}
        \end{center}
           \caption{Clip of the program in computation of $\pi$ under OpenMP framework.}
        \label{fig:long}
        \label{fig:onecol}
        \label{sample_code}
    \end{figure}
    \begin{table}
        \begin{center}
        \begin{tabular}{|c|c|c|}
        \hline
        number of threads & estimated value & consumed time\\
        \hline
        1 & 3.14159 &5.59387s \\
        2 & 3.14159 &2.69163s \\
        \hline
        \end{tabular}
        \end{center}
        \caption{Estimation Performance under different number of involved threads.}
        \label{estimation_performance}
    \end{table}
    \begin{figure}[t]
        \begin{center}
           \includegraphics[width=1.0\linewidth]{images/problem_statement.png}
        \end{center}
           \caption{The toy problem is formulated to a linked list of tasks.}
        \label{fig:long}
        \label{fig:onecol}
        \label{problem_statement}
    \end{figure}

    \subsection{Naive Reduction}
    \begin{figure}[t]
        \begin{center}
           \includegraphics[width=1.0\linewidth]{images/naive_implementation.png}
        \end{center}
           \caption{Implementation of naive conversion of the link list task.}
        \label{fig:long}
        \label{fig:onecol}
        \label{naive_implementation}
    \end{figure}

    A intuitive solution is to represent the orignal
    link list structure with array we are familiar with as shown in 
    Figure~\ref{naive_implementation}. This strategy requires us to 
    visit all the elements in link list and reorgnize 
    those elements to the array form. Once we obtain the array form, we 
    can solve this problem with the approach above. 

    
    \subsection{Result of Naive Implementation}
    Table~\ref{naive_serial} shows the result under 
    the naive implementation of reduction and pure serial 
    pattern. Obviously, this manual parallelization perform better 
    than the pure serial operation. But this strategy is not 
    scalable to other scenario since it requires specified nasty manual
    array structure building operation in a certain problem. In
    next section, we will introduce a new characteristic in
    OpenMP to tackle this problem. 

    \subsection{Reduction by Task}
    When we use the task command, we can solve this problem 
    in an elegant fashion as shown in Figure ~\ref{reduction_by_task}.
    Tasks are independent units of work, which are composed of
    code to execuate, data environment and internel control 
    variable. If we assign a segment of code as Task, the system 
    runtime will automatically create multiple tasks one for each
    thread. 
 

    \begin{table}
        \begin{center}
        \begin{tabular}{|c|c|c|}
        \hline
        Pattern &  consumed time\\
        \hline
        Serial Pattern & 10.840s \\
        Navie Reduction & 7.009s \\
        Task Reduction &  6.840s\\
        \hline
        \end{tabular}
        \end{center}
        \caption{Comparsion between different execuation patterns.}
        \label{naive_serial}
    \end{table}

    \begin{figure}[t]
        \begin{center}
           \includegraphics[width=1.0\linewidth]{images/reduction_by_task.png}
        \end{center}
           \caption{Clip of code to parallel the link list task with the task techinque.}
        \label{fig:long}
        \label{fig:onecol}
        \label{reduction_by_task}
    \end{figure}
    
    \section{Conclusion}
    In this project, we explore the basic usage of OpenMP function and 
    application of OpenMP in more complicated scenario. We achieve the 
    approximation of $\pi$ following the previous project but through
    integration under OpenMP framework in a neat fashion. Furthermore, we explore the
    usage of OpenMP in some advanced data structure such as link
    list instead of naive array data structure. Experiment shows that
    we can achieve parallelization over more complicated data structure 
    with task techinque under OpenMP framework. 

    {\small
    \bibliographystyle{ieee}
    \bibliography{egbib}
    }
    \begin{thebibliography}{1}

    \bibitem{IEEEhowto:kopka}
    % H.~Kopka and P.~W. Daly, \emph{A Guide to \LaTeX}, 3rd~ed.\hskip 1em plus
    % 0.5em minus 0.4em\relax Harlow, England: Addison-Wesley, 1999.
    % D. Pathak, R. Girshick, P. Doll ́ ar, T. Darrell, and B. Hariha-ran.
    % Learning features by watching objects move. In \emph{CVPR}, 2017.
    https://www.openmp.org/wp-content/uploads/Intro\_To\_OpenMP\_Mattson.pdf
    \end{thebibliography}

    \end{document}