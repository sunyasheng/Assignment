\documentclass[10pt,twocolumn,letterpaper]{article}

    \usepackage{cvpr}
    \usepackage{times}
    \usepackage{epsfig}
    \usepackage{graphicx}
    \usepackage{amsmath}
    \usepackage{amssymb}
    
    % Include other packages here, before hyperref.
    
    % If you comment hyperref and then uncomment it, you should delete
    % egpaper.aux before re-running latex.  (Or just hit 'q' on the first latex
    % run, let it finish, and you should be clear).
    \usepackage[pagebackref=true,breaklinks=true,letterpaper=true,colorlinks,bookmarks=false]{hyperref}
    
    \cvprfinalcopy % *** Uncomment this line for the final submission
    
    \def\cvprPaperID{****} % *** Enter the CVPR Paper ID here
    \def\httilde{\mbox{\tt\raisebox{-.5ex}{\symbol{126}}}}
    
    % Pages are numbered in submission mode, and unnumbered in camera-ready
    \ifcvprfinal\pagestyle{empty}\fi
    \begin{document}
    
    %%%%%%%%% TITLE
    \title{Technical Report for Fourth Project in Parallel Computing}
    
    \author{Yasheng Sun\\
    117020910076\\
    {\tt\small sunyasheng123@gmail.com}
    % For a paper whose authors are all at the same institution,
    % omit the following lines up until the closing ``}''.
    % Additional authors and addresses can be added with ``\and'',
    % just like the second author.
    % To save space, use either the email address or home page, not both
    }
    
    \maketitle
    %\thispagestyle{empty}
    
    %%%%%%%%% ABSTRACT
    \begin{abstract}
       This report concludes the fourth project in parallel computing. This projcet requires
       us to compute the highest and lowest temperature in some specified cities, which involves
       a classical mapreduce pattern -- Summarization Pattern. We generate 10000000 items in 5
       cities and obtain the statistics of highest and lowest temperature through MapReduce technique. 
       All my code is publicly available on my github
        \url{https://github.com/sunyasheng/Parallel-Computing}.

    \end{abstract}
    
    %%%%%%%%% BODY TEXT
    
    
    %-------------------------------------------------------------------------
    \section{Data Preparation}
    
    Downloading the history temperature of a certain city is a non-trivial thing, requring usage
    of API to some website. Unfortunately, some project has been out of date such as weatherData project[1].
    Therefore, we generate the data randomly to simulate this scenario without loss of generality.
    The statistics of our generated data is shown in Fig.~\ref{statistics}. We generate our data according to Gaussian
    distribution. The mean is set to 0 and the variance is set to 100. Temperature in 5 cities 
    is generated and each city contains 10000000 weather items. 

    
    \section{Methodology}

    \begin{figure}[t]
        \begin{center}
        % \fbox{\rule{0pt}{2in} \rule{0.9\linewidth}{0pt}}
           \includegraphics[width=1.0\linewidth]{data_statistics.png}
        \end{center}
           \caption{Statistics of the generated data.  The mean is set to 0 and the variance is set to 100.}
        \label{statistics}       
        \label{fig:long}
        \label{fig:onecol}
    \end{figure}
    In this section, the whole workflow of MapReduce is illustrated in Fig.~\ref{workflow}. 
    The task of Mapper is to select every item which contains a city name
    and its corresponding temperature. Then the output stream will be
    shuffled, sorted and passed to Reducer. Reducer will go through all
    the items and maintain the highest and lowest temperature for each 
    city. 

    It is worthy to mention that the number of weather data transportation  
    from mapper to reducer is huge. The inter-node transportation is usually
    expensive. To lower the amount of transportation, Combiner is introduced
    here to collect the highest and lowest temperature in one single data node.
    The Combiner is essentially a lcoal reducer. With this pre-reduction, 
    the transportation amount between different nodes can be decreased significantly.
    
    \section{Result}
    
    The output of this mapreduce task is shown in Fig.~\ref{output}. The highest
    and lowest temperature for each city is given here. 
    \begin{figure}[t]
        \begin{center}
        % \fbox{\rule{0pt}{2in} \rule{0.9\linewidth}{0pt}}
           \includegraphics[width=1.0\linewidth]{workflow.png}
        \end{center}
           \caption{\label{workflow}The MapReduce workflow.}
        \label{fig:long}
        \label{fig:onecol}
    \end{figure}

    \begin{figure}[t]
        \begin{center}
        % \fbox{\rule{0pt}{2in} \rule{0.9\linewidth}{0pt}}
           \includegraphics[width=1.0\linewidth]{output.png}
        \end{center}
           \caption{System output for the highest and lowest
           temperature in those 5 cities.}
        \label{output}
        \label{fig:long}
        \label{fig:onecol}
    \end{figure}
    
    From Fig.~\ref{system_info} we could know that the mapper task is 
    split to five nodes and the output stream will be passed to one reduce 
    node to finally summarize the result after map operation.
    \begin{figure}[t]
        \begin{center}
        % \fbox{\rule{0pt}{2in} \rule{0.9\linewidth}{0pt}}
           \includegraphics[width=1.0\linewidth]{task_info.png}
        \end{center}
           \caption{System information in MapReduce Task.}
        \label{system_info}
        \label{fig:long}
        \label{fig:onecol}
    \end{figure}

    Fig.~\ref{with_combiner} shows that the Combiner technique significantly
    reduces the amount of reduce input records and reduce shuffle bytes compared
    with the traditional routine without Combiner as is shwon in Fig.~\ref{without_combiner}.
    This improvement will play a more important role when it comes to a larger scale
    dataset.

    \section{Conclusion}
    We show the classical MapReduce paradigme -- Summarization Pattern by a
    minimum/maximax search problem in this project. A practical technique, 
    Combiner, is introduced here to lower the transportation amount between 
    different node. Future work will involve realtime data grabbing and reducing task
    under distributed computation framework.
    \begin{figure}[t]
        \begin{center}
        % \fbox{\rule{0pt}{2in} \rule{0.9\linewidth}{0pt}}
           \includegraphics[width=1.0\linewidth]{detail_info.png}
        \end{center}
           \caption{The reduce input records and shuffle bytes information with Combiner.}
        \label{fig:long}
        \label{fig:onecol}
        \label{with_combiner}
    \end{figure}    
    \begin{figure}[t]
        \begin{center}
        % \fbox{\rule{0pt}{2in} \rule{0.9\linewidth}{0pt}}
           \includegraphics[width=1.0\linewidth]{new_detail_info.png}
        \end{center}
           \caption{The reduce input records and shuffle bytes information without Combiner.}
        \label{fig:long}
        \label{fig:onecol}
        \label{without_combiner}
    \end{figure}    
    
    {\small
    \bibliographystyle{ieee}
    \bibliography{egbib}
    }
    \begin{thebibliography}{1}

    \bibitem{IEEEhowto:kopka}
    % H.~Kopka and P.~W. Daly, \emph{A Guide to \LaTeX}, 3rd~ed.\hskip 1em plus
    % 0.5em minus 0.4em\relax Harlow, England: Addison-Wesley, 1999.
    % D. Pathak, R. Girshick, P. Doll ́ ar, T. Darrell, and B. Hariha-ran.
    % Learning features by watching objects move. In \emph{CVPR}, 2017.
    \url{http://ram-n.github.io/weatherData}
    \end{thebibliography}

    \end{document}